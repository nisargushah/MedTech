The world has been riding high on a technological wave. It has entered every aspect of our life and also has influenced several life threatening procedures as well. We live in a fast paced world where everything is digitized. From spending hours in the local library to researching, we now have access to several pages of target data in just a click.  This has helped us save a lot of time and focus on the vital tasks at hand. 

One such industry that is highly time sensitive is the healthcare industry. Taking down vitals on an hourly patients seems like a mundane task. However, even a minute change in dosage or a difference of an hour can drastically effect the patient outcomes. There are several factors that technology could help improve time and efficiency drastically. While talking to several professionals in the health care field, our team realized several factors that shift focus of health care professionals from patient care to administrative tasks that could easily be automated. 

Although a noble field, it is important to realize after all hospitals are also a business and administrative  are equally important to keep the business running. The medical world definitely has its set of jargon which effect medicine and equipment costs that a non-medical professional could perhaps not interpret. Thus it was important to bridge the gap between medicine and accounting so healthcare specialists are focused on medical duties rather than routine administrative tasks. After talking to professionals at different stages of their career, we reviewed other EHR systems and decided to come up with solutions that help accelerate the process of administrative and management tasks in the health care field. 

EMR, Electronic Medical Records, had slow adoption in the US in 2009 \cite{BeldenJ.L.GraysonR.AndBarnes2009}. This was due to the fact that it lacked efficiency and usability of EMRs currently available. This can increase time and costs, hampering productivity. For one specific example, Epic EMR is known to not be user-friendly and requires training. Medical professionals noted that it takes a long time to bill or fill out information, which cuts time for the patient care. Epic EMR also held 54\% of patients in the US in 2015 \cite{Glaze2015}. If the time it takes for a doctor to enter medical records is effectively shortened, he/she would be able to visit more patients and improve health overall. On the business aspect, more patients go through, which means more money is spent either from them or from insurance companies, which adds to the business's profit.

Dr. Shawn Gieser, our sponsor, outlined this inefficiency and selected our team to work on this project. Currently, we do not have a customer, but we are communicating with medical students and medical professionals to aid us in this project.